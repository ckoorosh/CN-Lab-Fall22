\documentclass[12pt]{article}

\usepackage{graphicx}
\usepackage{xcolor}
\usepackage{indentfirst}
\usepackage[a4paper, total={6in, 8in}]{geometry}
\usepackage{hyperref}
\usepackage{fancyhdr}
\usepackage{xepersian}
\settextfont{B Nazanin}
\setlatintextfont{Times New Roman}

\definecolor{codegray}{gray}{0.9}
\newcommand{\code}[1]{\colorbox{codegray}{\texttt{\lr{#1}}}}

\begin{document}


%title page%
\begin{titlepage}
	\begin{center}
		\vspace{0.2cm}
		
		\includegraphics[width=0.4\textwidth]{sharif.png}\\
		\vspace{0.2cm}
		\textbf{ \Huge{آشنایی با \lr{DHCP}}}\\
		\vspace{0.25cm}
		\textbf{ \Large{آز شبکه - دکتر بردیا صفایی}}
		\vspace{0.2cm}
		
		
		\large \textbf{دانشکده مهندسی کامپیوتر}\\\vspace{0.1cm}
		\large   دانشگاه صنعتی شریف\\\vspace{0.2cm}
		\large   ﻧﯿﻢ‌سال اول ۰۱-۰۲ \\\vspace{0.10cm}
		\large{ گروه 8:}\\
		\large{\href{mailto:mehrshad.mirmohammadi@gmail.com}{مهرشاد میرمحمدی - 98109634}}\\
		\large{\href{mailto:parhaamsaremi@gmail.com}{پرهام صارمی - 97101959}}\\
		\large{\href{mailto:mofayezi.m@gmail.com}{محمدرضا مفیضی - 98106059}}\\
	\end{center}
\end{titlepage}
%title page%

\newpage

%pages header
\pagestyle{fancy}
\fancyhf{}
\fancyfoot{}
\setlength{\headheight}{59pt}
\cfoot{\thepage}
\lhead{آشنایی با \lr{DHCP}}
\rhead{\includegraphics[width=0.1\textwidth]{sharif.png}\\
		دانشکده مهندسی کامپیوتر
}
\chead{آز شبکه - گروه 8}
%pages header

\section*{مقدمه}
در این جلسه به معرفی پروتکل \lr{DHCP} \LTRfootnote{\lr{Dynamic Host Configuration Protocol}}  برای دریافت خودکار آدرس در شبکه می‌پردازیم.
\section{\lr{DHCP}}
\lr{DHCP} یک پروتکل کلاینت-سرور است که به طور خودکار برای یک میزبان پروتکل اینترنت (\lr{IP}) آدرس \lr{IP} و سایر اطلاعات پیکربندی مرتبط مانند ماسک زیر شبکه \LTRfootnote{\lr{subnet mask}} و دروازه پیش فرض \LTRfootnote{\lr{default gateway}} را فراهم می‌کند.
به عبارت دیگر، \lr{DHCP} به میزبان‌ها اجازه می‌دهد تا اطلاعات پیکربندی \lr{TCP/IP }مورد نیاز را از سرور \lr{DHCP} بدست آورند.

هر دستگاه در یک شبکه مبتنی بر \lr{TCP/IP} باید یک آدرس \lr{IP} منحصر به فرد برای دسترسی به شبکه و منابع آن داشته باشد. بدون \lr{DHCP}، آدرس‌های \lr{IP} رایانه‌های جدید یا رایانه‌هایی که از یک زیرشبکه به شبکه دیگر منتقل می‌شوند باید به صورت دستی پیکربندی شوند. آدرس‌های \lr{IP} رایانه‌هایی که از شبکه حذف می‌شوند هم باید به صورت دستی بازیابی شوند.

با \lr{DHCP}، کل این فرآیند به صورت خودکار و مرکزی مدیریت می‌شود. سرور \lr{DHCP} مجموعه‌ای از آدرس‌های \lr{IP} را نگه می‌دارد و هنگام راه‌اندازی در شبکه، آدرسی را به هر کلاینت فعال \lr{DHCP} می‌دهد. از آن‌جایی که آدرس‌های \lr{IP} به جای ثابت (به طور دائمی اختصاص داده‌شده) پویا هستند، آدرس‌هایی که دیگر استفاده نمی‌شوند به طور خودکار برای تخصیص مجدد به محموعه \lr{IP} بازگردانده می‌شوند.

برای راه‌اندازی یک سرور \lr{DHCP} باید ابتدا یک یا چند مخزن \LTRfootnote{\lr{pool}} \lr{IP} را تعریف کنیم. در \lr{IPv4} فرایند ارتباط دستگاه با سرور \lr{DHCP} شامل 4 مرحله است که در شکل \ref{fig:dhcp-session} نمایش داده شده است.

\begin{figure}[h!]
	\centering
	\includegraphics[width=0.3\columnwidth]{figs/DHCP_session.png}
	\caption{نمایشی از یک \lr{session} در پروتکل \lr{DHCP}}
	\label{fig:dhcp-session}
\end{figure}

\subsection{سناریو اول}
سناریو شکل \ref{fig:scenario-1} را در نظر بگیرید.
\begin{figure}[h!]
	\centering
	\includegraphics[width=0.5\columnwidth]{figs/s-1.jpg}
	\caption{شمای سناریو اول}
	\label{fig:scenario-1}
\end{figure}
در این سناریو ما یک سرور برای سرویس \lr{DHCP}، دو \lr{PC}، یک
\lr{Light Weight Access Point}
و یک روتر به عنوان \lr{Gateway} در نظر می‌گیریم.
به منظور استفاده از \lr{DHCP} از ابتدا به \lr{PC}ها آدرسی نمی‌دهیم اما به سرور \lr{DHCP} به صورت دستی آدرس می‌دهیم.

با رفتن به سرور و بخش \lr{Services} تنظیمات \lr{DHCP} و تعریف مخزن 
\lr{IP}
را انجام دهید. 
حالا می‌توانید با رفتن به یکی از \lr{PC}ها و انتخاب گزینه \lr{DHCP} در بخش
\lr{IP Configuration}
آدرس‌ها را دریافت کنید. همچنین می‌توانید در \lr{Light Weight Access Point} آدرس و مقدار \lr{primary controller} را با \lr{DHCP} دریافت کنید.

\subsection{سناریو دوم}
سناریو شکل \ref{fig:scenario-2} را در نظر بگیرید.
\begin{figure}[h!]
	\centering
	\includegraphics[width=0.5\columnwidth]{figs/s-2.jpg}
	\caption{شمای سناریو دوم}
	\label{fig:scenario-2}
\end{figure}

در این سناریو سه زیر شبکه \code{172.6.1.0} (سمت چپ) و \code{172.16.2.0} (سمت راست) و \code{10.0.0.0} (وسط) را تعریف می‌کنیم.
ابتدا باید تنظیمات مسیریابی واسط‌های روترها را برای اتصال زیرشبکه‌ها انجام دهید و مشابه سناریو قبل سرور \lr{DHCP} را به صورت دستی آدرس‌دهی کنید.

از آن‌جایی که در این سناریو دو زیرشبکه داریم باید دو مخزن آدرس در سرور \lr{DHCP} تعریف کنید.
تا این‌جای کار می‌توانید \lr{PC}های زیرشبکه \code{172.16.2.0} را آدرس‌دهی کنید اما آدرس‌دهی زیرشبکه \code{172.6.1.0} انجام نخواهد شد.
برای حل این مشکل باید با رفتن به تنظیمات روتر و حالت \code{config} وارد بخش واسط روتر شوید.
سپس با دستور \code{ip helper-address} و دادن آدرس سرور \lr{DHCP} تنظیمات را انجام دهید.

\subsection{سناریو سوم}
در این سناریو قصد داریم مطابق شکل \ref{fig:scenario-3} یک روتر را به یک سرور \lr{DHCP} تبدیل کنیم.
\begin{figure}[h!]
	\centering
	\includegraphics[width=0.5\columnwidth]{figs/s-3.jpg}
	\caption{شمای سناریو سوم}
	\label{fig:scenario-3}
\end{figure}

در ابتدا باید در روتر سرور، مخزن آدرس را به‌صورت دستی تنظیم کنید. برای این‌کار باید بعد از رفتن به حالت کانفیگ دستور \code{ip dhcp pool [pool-name]} را وارد کنید.

حالا باید دامنه آدرس‌های دلخواه را با دستور \code{network [network-number] [network-mask]} تعیین کنید (به‌عنوان مثال در این سناریو می‌توان \code{network 10.0.0.0 255.255.255.0} را وارد کرد).
در نهایت \lr{default gateway} را با دستور \code{default-router [router's ip address]} تنظیم کنید.

در روتر کلاینت، ابتدا وارد واسط مورد نظر شوید. حالا می‌توان با دستور \code{ip address dhcp} آدرس کلاینت را تنظیم کرد.
می‌توانید با دستورات \code{ip interface brief} و \code{show ip route} از صحت آدرس‌دهی مطمئن شوید.


\subsection{سوال‌ها}
\begin{enumerate}
	\item 
	روش‌های مختلف اختصاص \lr{IP} در پروتکل \lr{DHCP} را شرح دهید.
	\item 
	\lr{DHCP} از چه پروتکلی برای برقراری ارتباط استفاده می‌کند؟
	\item 
	پروتکل \lr{DHCP} در حالت عادی هیچ سازوکاری برای احراز هویت ندارد و در نتیجه در معرض حمله‌های مختلفی است. حمله‌های محتمل در این پروتکل را شرح دهید.
	
	
\end{enumerate}

\end{document}
